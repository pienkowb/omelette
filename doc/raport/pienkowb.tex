Poza pracą wynikającą z~pełnienia obowiązków kierownika projektu, skupiającą się na administrowaniu głównym repozytorium, sporządzaniu raportów i~dbaniu o~równy przydział pracy, aktywnie uczestniczyłem w~podejmowaniu wszystkich decyzji projektowych i~tworzeniu architektury programu. Miałem duży wpływ na ostateczny wygląd składni języka -- wiele z~pomysłów, które zaproponowałem, miało przełożenie na obowiązujący projekt gramatyki.

Samodzielnie zaprojektowałem moduł \texttt{code}, odpowiedzialny za wstępne dzielenie kompilowanego kodu na części zawierające pojedyncze obiekty. Ostatecznie potencjał tego modułu nie został w~pełni wykorzystany, gdyż kod za każdym razem kompilowany jest w~całości, a~napisana przeze mnie klasa \textbf{Code} pozwalała ustalić, które obiekty uległy zmianie, pod warunkiem aktualizowania kodu w~trakcie edycji.

Odpowiadam także za pierwszą wersję klasy \textbf{DrawableFactory}, która zapewniała dynamicznie ładowanie klas z~konkretnych modułów na podstawie typu obiektu UML. Zaletą takiego rozwiązania była możliwość rozszerzania języka o~kolejne elementy bez zmian w~module rysującym. Klasa ta została później udoskonalona przez innego członka zespołu i~stała się częścią modułu \texttt{diagram}, w którym przeprowadziłem gruntowną modyfikację zmieniając architekturę modułów z~klasami rysującymi.

Kolejnym moim dokonaniem jest klasa \textbf{Compiler}, której zadaniem jest przeprowadzanie kolejnych etapów kompilacji począwszy od parsowania, a na walidacji kończąc. Integruje ona kilka innych klas -- m.in. klasę \textbf{DependencyResolver} realizującą rozwiązywanie relacji dziedziczenia pomiędzy obiektami oraz klasę \textbf{Validator} odpowiedzialną za weryfikację zdefiniowanych kluczy obiektu na podstawie danych walidacji. W~przypadku obu tych klas miałem ogromny wpływ na ich projekt oraz implementację.
