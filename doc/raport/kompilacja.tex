% subsection

Kompilacja diagramów w programie \omlet składa się z następującego zestawu czynności:

\begin{itemize*}
  \item{Wczytanie kodu i jego wstępny podział na obiekty}
  \item{Sparsowanie kodu}
  \item{Uzupełnienie obiektów danymi z ich prototypów}
  \item{Odrzucenie prototypów}
  \item{walidacja skompilowanych obiektów}
\end{itemize*}

\subsubsection{Wczytanie kodu}
Za ten etap odpowiedzialna jest klasa \texttt{Code}.
Klasa przyjmuje napis, zawierający plik źródłowy.
Za pomocą klasy \texttt{Lexer} sprawdzane są kolejne linie.
Jeżeli linia jest definicją obiektu, tworzony jest nowy obiekt typu \texttt{\_CodeObject}, zawierający linie należące do obiektu następującego po znalezionym nagłówku.
Poza samymi liniami \texttt{\_CodeObject} przechowuje także informacje pozwalające określić numery linii w źródle.

Ponieważ pierwszy wiersz źródła nie musi być nagłówkiem obiektu, tworzony jest specjalny obiekt zerowy, przechowujące wszystkie linie przed pierwszym obiektem.

\subsubsection{Parsowanie kodu}
Obiekt klasy \texttt{Code}, zawierający obiekty \texttt{\_CodeObject} jest następnie przesyłany do klasy \texttt{Parser}.
Klasa ta zawiera funkcje budujące obiekty klasy \texttt{UMLObject}, na podstawie rozpoznanych symboli.
Obiekt klasy \texttt{Parser} Tworzy obiekt klasy \texttt{Lexer}, w którym zdefiniowana jest gramatyka.

Następnie metody klasy \texttt{Parser}, służące do budowania obiektów \texttt{UMLObject} są dodawane jako obiekty obsługujące symbole gramatyki.
Ostatecznie metoda \texttt{parse\_string} klasy \texttt{Lexer} zostaje wywołana i zostają utworzone obiekty \texttt{UMLObject} odpowiadające plikowi źródłowemu.

\subsubsection{Uzupełnienie obiektów}
Za uzupełnienie obiektów odpowiada obiekt klasy \texttt{DependencyResolver}.
Obiekt ten otrzymuje wszystkie obiekty \texttt{UMLObject}, które zostały stworzone w poprzednich krokach (oraz obiekty biblioteczne, tworzone przy inicjalizacji kompilatora).

Na początku obiekt sprawdza, czy nie występują cykliczne zależności pomiędzy obiektami (t.j. czy dwa lub więcej obiektów nie jest wzajemnie swoimi prototypami).

Następnie dla każdego obiektu znajdowany jest obiekt, będący jego prototypem i właściwości tego obiektu są dopisywane do aktualnego obiekt (pod warunkiem, że obiekt sam takich właściwości nie określa).

Po dojściu do obiektu, którego prototypem jest \texttt{base}, uzupełnianie danej gałęzi obiektów zostaje zakończone.

\subsubsection{Odrzucanie prototypów}
Prototypy muszą być odrzucone z puli obiektów z dwóch powodów:
\begin{itemize*}
  \item{Nie powinny być rysowane na diagramie}
  \item{Nie muszą być spójne} (spełniać wszystkich wymagań określonych przez ich prototypy). Pozostawienie ich przed następnym krokiem spowodowałoby niepotrzebne błędy.
\end{itemize*}

\subsubsection{Walidacja obiektów}
Obiekt klasy \texttt{Validator} sprawdza zgodność z danymi walidacji określonymi przez jego prototypy.

Normalnie sprawdzane jest istnienie, bądź nie istnienie konkretnej właściwości obiektu, oraz ew. zgodność typu tej właściwości.
Jeżeli prototypy określają wymagane wartości o typie \texttt{Object}, to sprawdzane jest także, czy wskazane przez te wartości obiekty istnieją.
