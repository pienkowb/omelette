\section{Interfejs programu}
Na samym początku projektu, po wybraniu jego tematu, planowaliśmy stworzyć narzędzie obsługiwanie z linii komend do "kompilacji" plików tekstowych w diagramy UML. Jednak z upływem czasu doszliśmy do wniosku, że łatwiejsze i wygodniejsze w użyciu będzie proste IDE. W pierwszej wersji okno apliakcji podzielone było na dwie części: część zawierającą kod i diagram. Diagram był statycznym obrazem generowanym na podstawie kodu w sąsiedniej części okna. Kolejnym krokiem było umożliwienie użytkownikowi przeciąganie elementów na diagramie, który przestał być statyczny. Również okno z kodem otrzymało nową funkcjonalność jaką było podstawowe kolorowanie składni opartej o syntaktykę języka Python. Dodano również przycisk pozwalający na wygenerowanie diagramu w postaci rastrowej i zapisanie jej do pliku. Okno kodu zostało wypsażone w obsługę składni stworzonego przez nas języka. Kolejnym etapem było wykorzystanie architektury paneli do budowy wewnętrznej struktury okna oraz dodanie panelu obsługi błędów przetwarzania kodu.
\section{Język opisu diagramów UML}
Od samego początku zakładaliśmy strukturę języka opartą o system klucz-wartość, którą utrzymaliśmy do końca trwania projektu. Naszym głównym celem było uzyskanie jak największej elastyczności języka, która miała umożliwiać definiowanie własnych elementów diagramów bez konieczności ingerencji w wewnętrzne struktury aplikacji. Zostało to uzyskane przez zastosowanie architektury modułów dołączanych dynamicznie przy uruchomieniu aplikacji. W obecnej formie możliwa jest walidacja kodu poddawanego kompilacji oraz zwrócenie wykrytych błędów wraz z numerami linii w których wystąpiły oraz prostą diagnostyką.
