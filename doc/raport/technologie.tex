\subsection{Python}

\begin{wrapfigure}{r}{0.22\textwidth}
  \begin{center}
    \includegraphics[width=0.2\textwidth]{python}
  \end{center}
\end{wrapfigure}
Jako język programowania wybraliśmy język \textbf{Python}.

\textbf{Python} to interpretowany język ogólnego zastosowania, pozwalający programować na wysokim poziomie abstrakcji.
Język oparty jest na wielu paradygmatach programowania: programowaniu obiektowym, programowaniu funkcjonalnym i programowaniu imperatywnym.
Typowanie w \textbf{Pythonie} jest dynamiczne, a typy są mocne.
Zarządzanie pamięcią odbywa się dynamicznie.

Interpretery \textbf{Pythona} dostępne są na wszystkie popularne systemy operacyjne, wiele z nich jest otwartym oprogramowaniem.
Sama specyfikacja języka zarządzana jest przez \emph{Python Software Foundation}~--- niezależną organizację non-profit.

Głównym powodem, dla którego zdecydowaliśmy się na język \textbf{Python} to jego popularność.
Język ten ma opinię języka o bardzo dobrej dokumentacji.
Popularność wpływa także na dostępność dużej ilości otwartych bibliotek, z których wiele jest dojrzałych i wysokiej jakości.

Wielu z nas korzysta na co dzień z systemu GNU/Linux, gdzie wiele aplikacji jest napisanych w języku \textbf{Python}.
Znajomość języka \textbf{Python} pozwoliłaby nam więc robić zmiany w aplikacjach, z których korzystamy na co dzień.

Jednym z powodów, dla którego wybraliśmy język \textbf{Python} był także fakt, że nikt z nas go nie znał.
Wybranie nieznanego dotąd języka miało sprawić, że projekt będzie bardziej interesujący oraz zwiększyć kompetencje zawodowe członków zespołu
\footnote{W razie gdyby projekt nie okazał się hitem na miarę Napstera i musielibyśmy się jeszcze kiedykolwiek starać o pracę.}
.

\subsection{pyparsing}
\textbf{Pyparsing} to jedna z bibliotek, które skłoniły nas do wybrania języka \textbf{Python} do realizacji tego projektu.
Biblioteka \textbf{Pyparsing} to otwarte oprogramowanie.

Biblioteka pozwala w prosty sposób zbudować rekursywny analizator składniowy zstępujący.
Gramatyka, pod kątem której analizowany ma być plik źródłowy, określana jest za pomocą języka \textbf{Python} w plikach źródłowych projektu (W naszym przypadku w pliku \texttt{lexer.py}).
Pyparsing jest używany w takich projektach jak Django, pydot czy Graphite.

Parsowaną gramatykę opisuje się tworząc odpowiednie obiekty.
Obiekty te mogą reprezentować symbole terminalne (wyrażenia regularne, zestawy znaków, pojedyncze znaki lub ich ciągi) lub ich produkcje.
Każdemu obiektowi można przypisać akcję, która zostanie wykonana, gdy dany symbol zostanie wczytany.

\subsection{Qt i pyQt}
\begin{wrapfigure}{r}{0.22\textwidth}
  \begin{center}
    \includegraphics[width=0.2\textwidth]{qt}
  \end{center}
\end{wrapfigure}
\textbf{Qt} to zestaw przenośnych bibliotek i narzędzi programistycznych dedykowanych do języków \emph{C++} i \emph{Java}.
Pozwala budować graficzne interfejsy użytkownika w sposób zorientowany obiektowo.

Środowisko Qt to otwarte oprogramowanie.
Środowisko dostępne jest na platformy \emph{X11}, \emph{Windows}, \emph{Mac OS X}, \emph{Haiku}, oraz na urządzeniach przenośnych opartych na Linuksie, \emph{Windows CE} i~\emph{Symbian}.

Biblioteki \textbf{Qt} oprócz obsługi interfejsu użytkownika, zawierają także niezależne od platformy systemowej moduły obsługi procesów, plików, sieci, grafiki trójwymiarowej (OpenGL), baz danych (SQL), języka XML, lokalizacji, wielowątkowości, zaawansowanej obsługi napisów oraz wtyczek.

Dzięki bibliotece \textbf{pyQT} mogliśmy skorzystać ze środowiska \textbf{Qt} z poziomu języka \textbf{Python}.

Do graficznego projektowania interfejsu użytkownika użyliśmy programu \textbf{Qt Designer}.

Zdecydowaliśmy się na środowisko \textbf{Qt} ze względu na jego popularność, dostępność na platformy mobilne (wierzymy, że znajomość tego środowiska będzie dla nas cenna w przyszłości), fakt że jest to środowisko w pełni zorientowane obiektowo, oraz dostępność dojrzałej biblioteki do języka \textbf{Python}.

\subsection{Graphviz}
Kiedy stało się jasne, że nie uda nam się napisać dobrego algorytmu rozkładającego elementy grafów, zaczęliśmy rozglądać się za gotową alternatywą.
Znaleźliśmy program \textbf{Graphviz}~--- otwarte narzędzie służące do wizualizacji grafów, które obsługuje wiele metod rozkładania grafów.

Sam program \textbf{Graphviz} przyjmuje grafy opisane za pomocą prostego języka i generuje obrazki.
Jednak twórcy tego programu udostępnili jego funkcjonalności w formie biblioteki.

Skorzystaliśmy z biblioteki \textbf{pygraphviz}, która oferuje możliwość wywoływania funkcji udostępnianych przez bibliotekę \textbf{Graphviz} z poziomu \textbf{Pythona}.
